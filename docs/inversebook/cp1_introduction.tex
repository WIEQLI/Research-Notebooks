\chapterimage{chap1.jpg} % Chapter heading image
\chapter{普适离散反问题}

\emph{反问题(inverse problem)}与\emph{正问题(forward problem, modelization problem, simulation problem)}相对,是基于已经存在的\emph{观测结果},逆推出产生如此观测结果的\emph{模型}。一个最大的不同是,正问题里确定模型时,观测结果有唯一解;反问题里确定观测结果时,模型没有唯一解。

故而,为了得到有效的解、或\emph{解集(collection of models)},往往须要对问题施加约束条件,这种约束条件可以看成是对模型的一种先验信息,它定义了解集可能存在的概率分布(因而我们可以探索到最大概率的解)。

当数据有较高的采样频率时,可以描述一个\emph{离散反问题(discrete inverse problem)}来代替原问题。离散反问题从数学上更容易解,但它不能完全等同于原问题,因为离散化的过程中引入了一些简化,造成一些信息因为离散化被隐藏了起来。

本章的核心是\emph{参数集(parameter set)}上的``\emph{状态信息(state of information)}'',一般被描述为前述的先验信息/概率密度。这种方法可以用在完全非线性问题里,但计算开销甚大。且它的解一般少于传统方法得到的解。

\section{模型空间与数据空间}
研究一个物理系统$\mathfrak{G}$一般分为三步:

\begin{enumerate}
  \item \textbf{\emph{系统参数化(parameterization of system)}}: 确定一个最小的参数集,使得通过这些参数可以完全描述物理系统;
  \item \textbf{\emph{正向模型(forward modeling)}}: 预测并给定一些参数,通过物理定律,建立参数$\rightarrow$观测结果之间的关系;
  \item \textbf{\emph{反向模型(inverse modeling)}}: 使用这些预测的参数,推测出实际模型参数的解。
\end{enumerate}

可见,前两步是\emph{归纳(inductive)},最后一步是\emph{演绎(deductive)},前两步的物理建模可能更困难;至于最后一步,有成熟、有效的数学方法来解。

首先让我们考量\emph{系统参数化}的问题:

\begin{example}
  考虑一个弹性系统$\mathfrak{G}$,对于空间上任意的点$\mathbf{x}$,有压力$\sigma^{ij}(\mathbf{x})$,拉力$\varepsilon^{ij}(\mathbf{x})$,则可以定义弹性硬度张量$c^{ij}_{kl}(\mathbf{x})$为:
  
  \begin{align} \label{fml:c1:stiffness}
    \sigma^{ij}(\mathbf{x}) = c^{ij}_{kl}(\mathbf{x}) \varepsilon^{kl}(\mathbf{x}).
  \end{align}
  
  同时还可以定义弹性柔度张量$s^{ij}_{kl}(\mathbf{x})$为:
  
  \begin{align} \label{fml:c1:compliance}
    \varepsilon^{ij}(\mathbf{x}) = s^{ij}_{kl}(\mathbf{x}) \sigma^{kl}(\mathbf{x}).
  \end{align}
  
  这里,张量$\mathbf{s}$是张量$\mathbf{c}$的逆,亦即$c^{ij}_{kl}s^{kl}_{mn}=\delta^{i}_{m}\delta^{j}_{n}$。故而选择这两个参数中的任何一个都是\emph{完全等价(completely equivalent)}的。
\end{example}

选择任意一组符合上述要求的特定参数集就是\emph{系统参数化},对于两个等价的参数集,其参数之间可以一一映射。

\subsection{模型空间} \label{sec:c1:modelspace}

与系统参数化无关,我们可以引入一个\emph{抽象空间},或称为\emph{流形(manifold)},用其中的每一个\emph{点}代表一个模型$\mathbf{m}$,于是这样的流形记为\emph{模型空间(model space)}~$\mathfrak{M}=\{\mathbf{m}_1,\mathbf{m}_2,\ldots\}$。

一个\emph{模型空间}可以被不同的(等价)系统参数描述。例如,设若$\mathfrak{M}$的参数有两维,记为$\{m^1, m^2\}$,则$m^1$和$m^2$可以选用不同的参数进行测量。由此定义的\emph{模型},或记为\emph{点}$\mathbf{m}$,一般来说不可以看成是一个(线性空间内的)向量,因为其坐标轴可能是非线性的。即使坐标轴都是线性的,也一般不是笛卡尔坐标系。

\begin{example} \label{exp:c1:dist}
  考虑一个\emph{模型空间}$\mathfrak{M}$已经确定了系统参数为$\{\kappa, \mu\}$(分别为体积模量和剪切模量),可以定义两个模型$\mathbf{m}_1,~\mathbf{m}_2$之间的距离为
  
  \begin{align} \label{fml:c1:m1m2dist}
    d = \sqrt{{\left(\log \frac{\kappa_2}{\kappa_1}\right)}^2+{\left(\log \frac{\mu_2}{\mu_1}\right)}^2}.
  \end{align}
  
  若以对数化的系统参数$\{\kappa^{\ast}=\log \kappa, \mu^{\ast}=\log \mu\}$来代替\eqref{fml:c1:m1m2dist}的参数,有:
  
  \begin{align} \label{fml:c1:m1m2distlog}
    d = \sqrt{{\left(\kappa^{\ast}_2 - \kappa^{\ast}_1\right)}^2+{\left(\mu^{\ast}_2 - \mu^{\ast}_1\right)}^2}.
  \end{align}
  
  可见,\eqref{fml:c1:m1m2distlog}是的参数是笛卡尔坐标系,但\eqref{fml:c1:m1m2dist}的不是。
\end{example}

\begin{character}[离散性] \label{cha:c1:modeldiscre}

须知的是,一个实际物理模型,其\emph{参数系(set of quantities)}(亦即坐标轴)往往是连续的,记为$m(\mathbf{x})$;但本章考虑的模型是经过抽样、离散化的模型,其\emph{参数系}为$m^{\alpha}=m^{\alpha}(\mathbf{x})$。坐标轴的离散性满足以下条件:

\begin{enumerate}
  \item 参数系由选取的系统参数决定,一个模型可以选用不同的参数系;
  \item 参数系维数$\alpha$有限,本章不考虑无限的情况;
  \item 参数系的抽样应该足够多,能不失真地反映出原始模型的特征;
  \item 参数系的值本身可以取连续也可以取离散,为了方便表达,可以默认认为取值总是连续的,因为离散的取值可以用\emph{狄拉克(Dirac)}函数表达。
\end{enumerate}

\end{character}

\begin{character}[线性] \label{cha:c1:modellin}
  
  设任意两个同属一个模型空间$\mathfrak{M}$,使用同一系统参数$m^{\alpha}$的模型$\mathbf{m}_1,\mathbf{m}_2$,其任意第$i$维的值均满足:
  
  \begin{align} 
    (\mathbf{m}_1+\mathbf{m}_2)^{\alpha} &= (\mathbf{m}_1)^{\alpha} + (\mathbf{m}_2)^{\alpha}, \label{fml:c1:modellinearity1} \\
    (\lambda \mathbf{m})^{\alpha} &= \lambda(\mathbf{m})^{\alpha}.\label{fml:c1:modellinearity2}
  \end{align}
  
  这样的的$\mathfrak{M}$称为\emph{线性模型空间(linear model space)},记为$\mathbb{M}$
  
\end{character}

\autoref{exp:c1:dist}使用的对数化系统参数$\{\kappa^{\ast}, \mu^{\ast}\}$是可以构成一个\emph{线性模型空间}。因此,此例中的距离$d$相当于该线性空间的\emph{\textit{Euclidean}范数}。

为了处理一个反问题,设模型空间$\mathfrak{M}$为全集,其任意一个子集记为$\mathcal{A}$,则可以令全集概率$P(\mathfrak{M})=1$,且其子集概率$P(\mathcal{A})\in [0,1]$。此处定义的概率,是通过一种特定的关联关系定义的,这种关系不受到空间是否是线性的影响。并且,能相应地根据该概率求出每个模型的概率密度分布。

\subsection{数据空间}

对于一个模型的观测结果,由于观测仪器的不同也可能产生诸多不同的结果,设全集$\mathfrak{D}$为所有可观测仪器产生的观测结果的集合,那么任何一组来自一个仪器的测量结果都可以看成它的元素$\mathbf{d}$;与$\mathfrak{M}$相似,$\mathfrak{D}$同样也可以看成是一个\emph{流形},$\mathbf{d}$可以看成是流形中的一个\emph{点}。

与\autoref{cha:c1:modeldiscre}和\autoref{cha:c1:modellin}相似,我们他讨论的数据空间也有类似的性质。离散性此处从略,在此我们介绍线性:

\begin{character}[线性] \label{cha:c1:datalin}
  
  在一个线性数据空间$\mathbb{D}$内,任意两组测量结果(来自两个仪器)$\mathbf{d}_1,\mathbf{d}_2$,其任意第$i$维的值均满足:
  
  \begin{align} 
    (\mathbf{d}_1+\mathbf{d}_2)^{\alpha} &= (\mathbf{d}_1)^{\alpha} + (\mathbf{d}_2)^{\alpha}, \label{fml:c1:datalinearity1} \\
    (\lambda \mathbf{d})^{\alpha} &= \lambda(\mathbf{d})^{\alpha}.\label{fml:c1:datalinearity2}
  \end{align}
  
\end{character}

\subsection{合并流形}

有时候,一个物理系统的输入和输出参数可以清楚地分开,这时使用两个\emph{流形}~$\mathfrak{M},\mathfrak{D}$来描述并无问题。但某些情况下(例如有反馈系统),不能做到这一点。因此有时我们也用一个合并的\emph{流形}$\mathfrak{X}$来表示整个\emph{参数空间(parameter space)}。这种情况下,其元素(或\emph{点})$\mathbf{x}$被称为一组参数。

\section{信息态}

本节讨论了基于\emph{概率论}定义的\emph{信息态(states of information)}。

\subsection{概率的定义}

考虑一个有限维流形$\mathfrak{X}$,正如在\autoref{sec:c1:modelspace}末尾提到的那样,可以从中定义任意的子集$\mathcal{A},\mathcal{B},\ldots$,我们称$\mathfrak{X},\mathcal{A},\mathcal{B},\ldots,\varnothing$所属的为一个域$F$,且$F$是一个\emph{$\sigma$域($\sigma$Field)}。

\begin{definition}[$\sigma$域]
  由集合$\Omega$的众子集(例如$\mathcal{A},\mathcal{B},\ldots$)构成的\emph{超集(super set)},在满足以下条件时称为一个\emph{域(Field)}。
  
  \begin{enumerate}
    \item $\Omega \in F$;
    \item 若$\mathcal{A} \in F$,则$\bar{\mathcal{A}} \in F$;
    \item 若$\mathcal{A},\mathcal{B} \in F$,则$\mathcal{A} \cup \mathcal{B} \in F$。
  \end{enumerate}
  
  特别地,如果$F$内所有满足\emph{两两互斥}的集合$\mathcal{A}_i,i=1\rightarrow\infty$,其交集$\mathcal{A}$满足
  
  \begin{align} \label{fml:c1:sigmafield}
    \mathcal{A} = \bigcup_{i=1}^{\infty}\mathcal{A}_i,~~\mathcal{A} \in F,
  \end{align}
  
  则这样的\emph{域}$F$称为\emph{$\sigma$域}。
\end{definition}

由于满足\emph{$\sigma$域},若对每个子集关联到一个概率算子$P(\cdot)$,则该算子满足\autoref{cha:c1:probfield}。

\begin{character}[概率算子的基本性质] \label{cha:c1:probfield}
  \mbox{{}}
  
  \begin{enumerate}
    \item 互斥的集合$\mathcal{A},\mathcal{B}$满足: $P(\mathcal{A}\cup\mathcal{B}) = P(\mathcal{A}) + P(\mathcal{B})$;
    \item 趋向空集$\varnothing$的集合序列$A_1 \supseteq A_2 \supseteq \ldots$对应地满足$\lim\limits_{i\rightarrow\infty}P(A_i) = 0$;
    \item 空集$\varnothing$满足$P(\varnothing) = 0$;
    \item 任意的两集合并集的概率: $P(\mathcal{A}\cup\mathcal{B}) = P(\mathcal{A}) + P(\mathcal{B}) - P(\mathcal{A}\cap\mathcal{B})$.
  \end{enumerate}
\end{character}

于是,考量模型$\mathbf{x}\in\mathfrak{X}$(同时也是一组参量),则可以定义

\begin{definition}[模型集的概率]
  定义一个多维参量的实函数$f(\mathbf{x})$,则集合$\mathcal{A}$的概率定义为:
  
  \begin{align} \label{fml:c1:probmodelset}
    P(\mathcal{A}) = \int_{\mathbf{x}\in\mathcal{A}}{f(\mathbf{x})\mathrm{d}\mathbf{x}}.
  \end{align}
  
  若$\forall~\mathbf{x},~f(\mathbf{x})>0$且$P(\mathfrak{X})=1$,这样的实函数$f$称为概率密度函数,$P$称为概率或对\emph{标准化测度(normalized measurement)}的概率。
  
  若以上条件不满足,这样的实函数称为\emph{相对概率密度函数(relative probability density function)},$P$称为\emph{相对概率}或对\emph{非标准化测度(nonnormalized measurement)}的概率。
\end{definition}

例如,对定义域$\{x>0\}$的单变量实区间定义$f=1/x$,显然该函数对全集的积分$P$为$\infty$,故其不能代表概率,但可以用来比较两个区间$\mathcal{A},\mathcal{B}$上对应的积分$P$,这就是其称为\emph{相对概率}的原因。

由于模型$\mathbf{x}$往往有实际的物理意义(例如每个变量都代表一个物理量,具有一个特定的物理\emph{量纲(physical dimensions)}),由于$P$无量纲,通常$f$需要是一个有量纲的函数,并且其量纲可以抵消模型$\mathbf{x}$的量纲。

若对模型$\mathbf{x}$能进行一个坐标变换,使得新坐标系下的模型$\mathbf{x}^{\ast}=\mathbf{x}^{\ast}(\mathbf{x})$,于是有

\begin{align} \label{fml:c1:probchangecoor}
  P(\mathcal{A}) = \int_{\mathbf{x}\in\mathcal{A}}{f(\mathbf{x})\mathrm{d}\mathbf{x}} = \int_{\mathbf{x}^{\ast}\in\mathcal{A}}{f^{\ast}(\mathbf{x}^{\ast})\mathrm{d}\mathbf{x}^{\ast}} = \int_{\mathbf{x}^{\ast}\in\mathcal{A}}{f(\mathbf{x})\left|\frac{\partial \mathbf{x}}{\partial \mathbf{x}^{\ast}}\right|\mathrm{d}\mathbf{x}^{\ast}}.
\end{align}

\subsection{概率的解释}

\subsection{概率的性质}