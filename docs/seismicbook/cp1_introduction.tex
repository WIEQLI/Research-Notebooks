\chapterimage{chap1.jpg} % Chapter heading image
\chapter{引论}

石油勘探开始于\emph{地震探测(Seismic Soundings)}。由于岩石的体积远大于碳氢化合物的体积,油气的存在几乎不能影响到反射结果。但是,通过判断地层的形状,有丰富的经验可以推测出油气的位置。鉴于\emph{地震探测}成本比实际开采低得多,它的应用价值随着技术的进步越加显明出来。

该领域的研究主要基于以下几点:

\begin{enumerate}
  \item \textbf{数据的重复性}: 数据呈高密度的磁信号窄带分布(magnetic tapes),这些数据可以经过大量可重复实验。
  \item \textbf{计算机作为成像设备}: 计算机图像受数学、物理学和统计学理解的限制。但(1984年)现有的计算机容量完全可以应对地震波。
\end{enumerate}

\section{我们要讨论的问题}
考虑到如下形式的问题(注意参数和输出均有可能是矩阵)。

\begin{problem}[普适优化问题] \label{def:c1:common}
  \begin{subequations}
  	\renewcommand{\theequation}
  	{\theparentequation-\arabic{equation}}
  	\begin{align}
  	  \min~&f_0 (x),\label{fml:c1:common}\\ 
  	  s.t.~&f_i (x) \leqslant b_i,~i=1,~\ldots,~m. \label{fml:c1:commonconst}
  	\end{align}
  \end{subequations}
\end{problem}

如果对任意的$i=0,~1,~\ldots,~m$,均有

\begin{align} \label{fml:c1:commonconvex}
	f_i(\alpha x + \beta y) \leqslant \alpha f_i(x) + \beta f_i(y).
\end{align}

这样的问题称为\emph{凸(Convex)问题},如果\eqref{fml:c1:commonconvex}中的不等号方向反向,这样的问题就称为\emph{凹(Concave)问题}。凸/凹问题本质上都可以用相同的方式解出,特别地,如果\eqref{fml:c1:commonconvex}中的不等号退化为等号,这样的问题就称为\emph{线性问题}。

\section{两个经典问题}

\subsection{最小平方问题}
\label{c1:minsquare}

考虑存在列向量$x$,一个最小平方问题可以描述为:

\begin{problem}[最简的最小平方问题] \label{def:c1:minsquare}
	\begin{align} \label{fml:c1:minsquare}
		\min\limits_{\mathbf{x}} f_0(\mathbf{x}) = \min\limits_{\mathbf{x}} {\lVert \mathbf{A}\mathbf{x} - \mathbf{b} \rVert}^2_2.
	\end{align}
\end{problem}

如此,只需对$f_0$求取梯度,即可得到所需的最优解$x^{\ast}$,有如下推导

\begin{equation}\label{fml:c1:reviseA}
	\mathbf{A} = \left[
	\begin{array}{c}
		 \mathbf{a}^T_1 \\
		 \mathbf{a}^T_2 \\
		 \vdots \\
		 \mathbf{a}^T_n \\
	\end{array}
	\right].
\end{equation}

\begin{problem}[较一般的线性规划问题] \label{def:c1:minlinear}
	\begin{subequations}
		\renewcommand{\theequation}
		{\theparentequation-\arabic{equation}}
		\begin{align}
		\min\limits_{\mathbf{x}}~&\mathbf{c}^T\mathbf{x},\label{fml:c1:minlinear}\\ 
		s.t.~&\mathbf{a}_i^T\mathbf{x} \leqslant b_i,~i=1,~\ldots,~m. \label{fml:c1:minlinearconst}
		\end{align}
	\end{subequations}
\end{problem}